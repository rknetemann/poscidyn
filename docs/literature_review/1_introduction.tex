\chapter{Introduction}
\label{chap:introduction}

The behavior of vibrating systems has been observed and studied for multiple centuries. While early investigations were largely theoretical, the practical impact of vibration analysis began to emerge in engineering and science during the 19th century. In the 20th century, advances in mathematics, materials, and computational tools led to a rapid expansion of applications, making vibration theory an essential component in fields such as structural engineering, automotive design, aerospace, and acoustics. %CITATIONS!

In parallel with these developments, a new frontier in science and engineering began to unfold: the ability to understand and manipulate matter at microscopic and nanoscopic scales. This led to the emergence of micro- and nanoscience—the study of systems at the micrometer ($10^{-6}$ m) and nanometer ($10^{-9}$ m) scales. These disciplines have enabled breakthroughs in microfabrication, precision sensing, mechanical signal processing, and the development of highly miniaturized, high-performance devices. %CITATIONS!

Among the most promising outcomes of this convergence between vibration theory and nanoscale science is the development of nanomechanical resonators.

\section{Nanomechanical Resonators}
\label{sec:nanomechanical_resonators}

Nanomechanical resonators (NMR) are vibrating structures operating at the nanoscale, typically fabricated from materials such as silicon, carbon nanotubes, or graphene \cite{xu_nanomechanical_2022,greenberg_nanomechanical_2012}. Due to their small mass, high stiffness, and minimal energy dissipation, these devices exhibit exceptional properties, including high resonance frequencies, low power consumption, and extremely high quality factors. %CITATIONS!

Their sensitivity to external forces and masses at the molecular or atomic scale makes them ideal candidates for a wide range of applications, including mass sensing \cite{hanay_single-protein_2012,chen_performance_2009,chiu_atomic-scale_2008}, force detection \cite{moser_ultrasensitive_2013,fogliano_ultrasensitive_2021}, temperature sensing \cite{chen_performance_2009}, signal processing at high frequencies \cite{gouttenoire_digital_2010, jensen_nanotube_2007} and many more. As such, nanomechanical resonators form a crucial building block in the intersection of nanoscience, mechanics, and sensor technology.

Understanding the dynamics of these resonators is essential for many of their applications, as it allows for the inference of physical quantities such as added mass, applied force, or temperature changes. Although significant research has been conducted on modeling their dynamic behavior, accurately identifying the parameters that govern these dynamics remains a major challenge. This difficulty arises in particular under conditions of high-frequency operation or strong excitation, where nanomechanical resonators often exhibit pronounced nonlinear behavior. %CITATIONS!

\section{Identifying Nonlinear Dynamics}
\label{sec:identifying_nonlinear_dynamics}
\noindent
Test